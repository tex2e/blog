
\documentclass[dvipdfmx]{standalone}
\usepackage[T1]{fontenc}
\usepackage{newtxtext, newtxmath}

\usepackage{tikz} % xcolor -> graphicx -> tikz
% \usetikzlibrary{fit}
% calc, positioning, quotes, topaths, scopes, spy
% matrix, graphs, graph.standard, trees, chains, automata, mindmap, er, calendar
% shapes.(geometric|symbols|arrows|multipart|callouts|misc)?
% arrows, patterns, fadings, shadings, shadows, backgrounds
% circuits.logic.US, circuits.ee.IEC, lindenmayersystems, folding, petri, svg.path
% decorations.(pathmorphing|pathreplacing|markings|footprints|shapes|text|fractals)?
% datavisualization, datavisualization.formats.functions
% intersections, plothandlers, plotmarks, through

% \begin{document}
%   \begin{tikzpicture}[> = latex]
%     \tikzset{
%       int/.style={draw, fill=blue!20, minimum size=2em},
%       init/.style={pin edge={to-,thin,black}},
%     }
%     \node [int, pin={[init]above:$v_0$}] (a) {$\frac{1}{s}$};
%     \node (b) [left of=a, node distance=2cm, coordinate] {a};
%     \node [int, pin={[init]above:$p_0$}] (c) [right of=a, node distance=2cm] {$\frac{1}{s}$};
%     \node [coordinate] (end) [right of=c, node distance=2cm]{};
%     \path[->] (b) edge node[above] {$a$} (a);
%     \path[->] (a) edge node[above] {$v$} (c);
%     \draw[->] (c) edge node[above] {$p$} (end) ;
%
%   \end{tikzpicture}
% \end{document}

\usepackage{xcolor}

\begin{document}
  \small
  \scalebox{0.8}{
  \begin{minipage}{0.35\hsize}
    \begin{table}
      \centering
      % \caption
      {$\mathbb{Z}/4\mathbb{Z}$の乗算表}
      \begin{tabular}{c|ccccc}
        $\times$ & 0 & 1 & 2 & 3 \\
        \hline
        0        & 0 & 0 & 0 & 0 \\
        1        & 0 & {\color{red}1} & 2 & 3 \\
        2        & 0 & 2 & 0 & 2 \\
        3        & 0 & 3 & 2 & {\color{red}1} \\
      \end{tabular}
    \end{table}
  \end{minipage}
  \begin{minipage}{0.35\hsize}
    \begin{table}
      \centering
      % \caption
      {$\mathbb{Z}/5\mathbb{Z}$の乗算表}
      \begin{tabular}{c|ccccc}
        $\times$ & 0 & 1 & 2 & 3 & 4 \\
        \hline
        0        & 0 & 0 & 0 & 0 & 0 \\
        1        & 0 & {\color{red}1} & 2 & 3 & 4 \\
        2        & 0 & 2 & 4 & {\color{red}1} & 3 \\
        3        & 0 & 3 & {\color{red}1} & 4 & 2 \\
        4        & 0 & 4 & 3 & 2 & {\color{red}1} \\
      \end{tabular}
    \end{table}
  \end{minipage}
  }
\end{document}
